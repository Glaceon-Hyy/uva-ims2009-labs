\documentclass[a4paper,11pt]{article}

\title{Intelligent Multimedia Systems\\Mean-Shift Object Tracking}
\author{J. v. Turnhout (0312649) \& R. Tobi (0448710)}

\usepackage{amsmath}
\usepackage{amsfonts}
\usepackage{graphicx}
\usepackage{float}
\newcommand{\tbf}{\textbf}
\newcommand{\ds}{\displaystyle}
\newcommand{\ra}{\rightarrow}

%% To complete this course you must pass the exam and hand in a report
%% describing the work you did for this lab. Together with this report
%% you'll hand in your code and some illustrative result videos and still
%% images. Instead of mailing multi-gigabyte video documents, you should
%% put them somewhere in your web site (public_html) and  the link to me.
%%
%% Note that the most important part is to implement the tracker in a correct
%% way. After you made sure your implementation is correct, try the tracker on
%% a video in a different domain (search on the internet to find a good video
%% or create your own video). Keep in mind that when looking for a suitable
%% video, the goal is not to show how good the tracker works, but try to put
%% your finger on the strong and weak parts of the algorithm. After you analyzed
%% these strong and weak points of the algorithm, try to come up with suggestions
%% to improve your design. If there is still enough time, you can implement one
%% or more of your suggestions and analyze the results (for instance, how does
%% your improvement compare to the original algorithm with respect to the weak
%% as well as the strong parts of the original algorithm). If you run out of
%% time, you can describe how you would change the design and explain your
%% expectations of this change. 
%%
%%
%% The report should be about 10 pages long. This should be long enough
%% to give a thorough description of your work, including introduction,
%% conclusions and a discussion. In a paper-style report, the introduction
%% would consist of a problem description (object tracking) and some related
%% work (mention some approaches taken by other researchers, you can use
%% the reader for this or look some up yourself). Then, give an outline
%% of the approach that you have taken (mean-shift and color histograms).
%% Describe the used algorithm and features in you own words. In the next
%% sections, you can write about your implementation and the experimental
%% results. Make sure to mention all relevent issues that you have done
%% and why. For instance, make mention of your approach to color, which
%% color model and why. When writing this part, keep in mind that a reader
%% should be able to reproduce your results by using the information in this
%% report. After that, give a detailed report of the results of your tracker,
%% including graphs and screenshots. In the final section(s), you can give a
%% discussion of your results, in which you can comment on the tracking results
%% and introduce improvements that you have thought of. Furthermore, describe
%% what the effects of your improvement are, or, for lack of time, describe
%% what you expect the effects of the proposed improvements would be. Mention
%% things like processing speed (and possible improvements), the handling of
%% different domains (is one general solution possible, should the tracker be
%% adapted for each domain, how will the adaptation be done, etc.). Don't forget
%% to include your references at the end, and cite your sources in the text.

\begin{document}
	\maketitle

	\abstract{
		\noindent
		We implement a mean-shift tracker in \verb|MATLAB| for following objects in
		video and evaluate it on three different domains. The tracker's theoretical
		underpinnings are described and we explore weaknesses in its design.
	}

	\section*{Introduction}
		Robust real-time tracking of objects in a sequence of video frames is one of
		the core problems of computer vision. A target object needs to be localized
		and represented in such a way as to be insensitive to changes in illumination,
		rapid movement, occlusion conditions, cluttering, background appearance, and
		scale. Furthermore, the per-frame computations should take no more than $100$
		milliseconds to be useful for real-time applications. These challenging
		conditions have led to many different approaches, most probabilistic in
		nature.
		\\ \\
		To track an object, we must iteratively establish its position in subsequent
		frames based on a known initial position. The essential step is then to create
		a target representation, such as a histogram, and attempt to find the position
		that best matches this representation in the next frame(s). If the tracker
		employs a brute-force algorithm, we may only exploit temporal locality in
		this matching step; that is, we assume the object's new position will be
		``close to'' its old one and search outwards in eg. a radial pattern. Such
		a brute-force tracker does no more than comparing the histogram $h_{x,y}^{w,h}$
		centered at each pixel $(x, y)$ with search-window dimensions $(w, h)$ to
		the target object histogram $h_{O}$ specified beforehand. Although this
		approach can be parallelized, the computational complexity renders it
		generally infeasible for larger windows.
		\\ \\
		A more efficient strategy is to incorporate kernel-weighted histograms.
		Each histogram pixel is now assigned a weight by a kernel mask based on
		its distance from the center-point. Because pixels further from the
		center of the tracked object will be the first to become occluded or
		warped under transformation, their values are less reliable. A kernel
		also imposes a ``smooth'' distance metric over histograms, which is
		required for the mean-shift step. The target's center pixel acts as the
		reference point; thus, the window width and height need to be odd.
		The efficiency gain comes from the fact that we can now steer the
		tracker towards the object's most likely new position by following
		the gradient rather than by examining all pixels within the local
		environment. This new position is derived from the histogram with
		smallest distance to the target's histogram representation.
		\\ \\
		%% The mean-shift tracker is a special case of the kernel-based tracker.
		%% Instead of doing a brute force search of all surrounding pixels, a gradient
		%% method is used to efficiently move towards the new position of the target.
		%% The new position of the target will of course be the positions of the candidate
		%% histogram with the smallest distance from the target. But we won't calculate
		%% the distance for all pixel positions, we will do it in an iterative fashion,
		%% starting from the last known position.
		KOEN:
		The mean-shift algorithm requires that we use a certain class of kernels for
		our kernel based histograms, only for this class the algorithm has been proved
		to converse (in most cases). So we need to calculate the Epanechnikov kernel.
		The formula is in the paper, but you will need these constants: $d = 2, c_d = \pi$.
		In my implementation I calculate the kernel beforehand and store the values in a
		matrix the size of the target region. This way I can look up the weight of a pixel
		with one look in the kernel matrix.
		\\ \\
		The mean-shift step looks quite complicated, but is essentially easy to implement,
		once you know what is going on. It calculates the step that has to be taken from
		the current position towards the new position. The next step in the mean-shift
		algorithm is a little trick to compensate for too large mean-shift steps. Occasionally
		the calculated step is too large, the next step would of course move (back) towards
		the optimal position, but mean-shift calculations are expensive. Therefore it is
		efficient to correct for the overshoot and this step does just that. It picks a
		point between the previously and newly calculated position and sees if that point
		is a better match (compare histograms) than the calculated position. If it is the
		position half-way is chosen and the process is repeated. This will correct overshoot
		without needing another mean-shift step.
		\\ \\
		The basis of the mean-shift tracker is a kernel based histogram creation function.
		You will need to implement this. This function takes a position or offset as parameter
		and will then calculate the histogram for an area of width x height pixels. This means
		finding the proper bin for each pixel and looking up the weight from the kernel function
		(see Lab 8), this weight is then added to the calculated bin. This function needs to be
		completed before you can start on the mean-shift tracker. Note: these kernel weights
		mentioned here are values of the kernel function k(). Assuming that this weight is w
		would be a mistake.
		\\ \\
		For the tracking fase you calculate the kernel based histogram for the target. Then
		for each new frame you calculate the histogram at the last known position of the target,
		this is the candidate histogram. The kernel and histogram are centred around the position.
		For the mean-shift step you will calculate for a region of width x height pixels, centred
		on the candidate position a combination of formula 10 and 13 from the paper to get a new y.
		\\ \\
		After performing this formula, you have the mean-shift, the amount you need to adapt your
		position to move closer to the target. The following step in the paper is sort of a hack
		to compensate for too large shifts. The algorithm sometimes calculates a step that will
		move beyond the target. Subsequent mean-shift steps will lead to the optimal position,
		but it is computationally cheaper to solve this in another way: See if the position halfway
		between the calculated new position and the original position is a better match (Bhattacharyya),
		if it is: take this position as new position and repeat.
		\\ \\
		One final issue regarding the processing of the frames is this. In lab 6, I gave a little
		piece of code to read the frames subsequently. The whole idea of tracking is, that you try
		to track a player in MULTIPLE frames. So, do not stop after one or two frames, but see how
		your algorithm handles several frames in a row. Ideally, you would process all subsequent
		frames, but if this takes too long, you could consider to skip some (one or two) frames
		after processing a frame, eg. after processing frame x, continue with frame (x+1); however,
		if it takes too long, then skip frame (x+1) and continue with frame (x+2). However, please
		keep in mind that tracking does not stop after one frame! See what happens if you track a
		player over several seconds of video.
		\\ \\
		After processing the frames, it would be nice to see the results of your tracking algorithm.
		You can do this by painting a rectangle in a frame on the new position and write this image
		to disk. However, Matlab also provides the possibility to create an avifile from your frames.
		The matlab-function to do this is called \verb|avifile|. 

	\section*{Theoretical Background}
		Two components can be distinguished in a tracker; the \textit{Object Model}
		and the \textit{Tracking Model}. The first is concerned with localizing and
		representing the type of object to be tracked up from the individual pixels,
		while the second deals with the high-level task of filtering to describe object
		motion.
		%% FIXME: TOO VAGUE?
		In abstract terms, this process of filtering can be modelled as a discrete-time
		dynamical system through the well known state-space approach. Each state $x_k$
		(for $k=0, 1, ...$) characterizes the relevant information about the tracked
		object, and dynamically evolves over time according to $x_k = f_k(x_{k - 1}, v_k)$
		where $v_k$ is the $k$-th (i.i.d) noise item. The states are not directly observable;
		rather a sequence of measurements $\{z_k\}$ evolving via $z_k = h_k(x_k, n_k)$
		(with $\{n_k\}$ another i.i.d noise sequence) provides ``hints'' about the true
		state at each time-step $k$, which is a latent variable. The tracker's task
		is now to estimate $x_k$ given all preceding measurements $z_{1:k}$, ie. it
		must recover the unknown probability density function $P(x_k \mid z_{1:k})$.
		This formulation leads to the general technique of recursive Bayesian estimation,
		which assumes the states $\{x_k\}$ and observations $\{z_k\}$ to be the hidden
		and visible parts of a hidden Markov model (HMM). The RBE method constructs the
		p.d.f in a prediction and an update step conceptually similar to those of the
		Expectation-Maximization algorithm family. When $v_k$ and $n_k$ are Gaussian
		and the prediction and update functions $f_k$ and $h_k$ are (non-)linear, the
		estimator becomes an instance of the (Extended) Kalman filter. However, other
		filter classes can also be used.
		\\ \\
		The mean-shift tracker assumes only small changes in location and appearance
		of the target object between two consecutive frames. This enables an efficient
		gradient-based localization scheme. Specifically, the target object is spatially
		masked with an isotropic kernel, which allows defining a smooth similarity function
		and reducing the localization problem to a search in this function's basin of attraction
		(ie., all points in the state-space that enter the same region $R$ over time). The MST's
		object model is given by a histogram in some color-space\footnote{We evaluate our tracker
		in \verb|RGB|, \verb|rgb|, \verb|HSV|, and \verb|OSC| for a fixed number of bins per
		dimension.}.
		%%
		%% TODO: mention color-spaces (RGB, nRGB, HSV, OSC)
		%% TODO: algorithm pseudo-code overview like in the ELPL report

	\section*{Implementation}
		The Epanechnikov weight-function is implemented by pre-calculating a 2D
		table of kernel values equal in size to the target region, which stores
		the weight for each relative pixel position. We add the weight per pixel
		to the corresponding histogram bin via a single lookup into this table.
		\\ \\
		Since caching all the frames of our animation test-sequences prior to
		tracking would cost a prohibitive amount of memory, we read each frame
		incrementally from disk on an as-needed basis. However, this reduces
		performance. Ideally, for any given frame number, we would pre-load the
		next $n$ frames in a separate thread, but \verb|MATLAB| provides no such
		facilities.
		\\ \\
		Histogram distances
		\\ \\
		To check the result you get from the object detection and to maybe choose
		between several detected objects, you can compare the histograms of the
		target and the candidate objects. So you create a histogram for the candidate
		objects in exactly the same fashion as you did for the target. Now you need
		to compare the candidate and target histogram, in some meaningful way. The
		reader (available on Blackboard and for sale in paper form as well) discusses
		several methods. I will repeat two here:
		\\ \\
		Euclidean distance: The Euclidean distance between histograms calculates the
		following distance between all histogram bins: $(b1 - b2)^2$, then sums this
		over all bins and takes the square root.
		\\ \\
		Bhattacharyya distance: The Bhattacharyya distance is a measure to compare statistical
		distributions, something that our histograms represent. The Bhattacharyya distance is
		calculated as follows. First the Bhattacharyya coefficient is determined by
		$\rho[p, q] = \int \sqrt{p(u) \cdot q(u) } \delta u$, then the Bhattacharyya
		distance is determined from this by $d = \sqrt{1 - \rho[p, q]}$.

	\section*{Evaluation and Results}
		KOEN: If you think your tracker is finished, you can think about how to evaluate
		the performance of your tracker. Further, you can think about the strong and
		weak parts of your tracker, and propose possible improvements. If there is
		enough time, you can implement one or more of your proposed improvements and
		analyse the effect compared to the original tracker.
		\\ \\
		Once you think you completed the mean-shift tracker, it is time to test your
		implementations on videos from different domains and analyze the performance
		of different color spaces. What are the advantages and disadvantages of certain
		color spaces, when do you think you should use one color space instead of the
		other, etc. After that, you can think about improvements of the tracker. This
		involves a thorough analysis of your current version, including an analysis of
		the strong and weak points. If you know when your tracker goes wrong, you can
		think of possible solutions to overcome these difficulties.
		\\ \\
		However, first think about how to evaluate the performance of your tracker. It
		would be nice to show some quantitative results in your report, instead of pointing
		at the videos and stating ``Look, it works!''. Several options exist, some more
		time-consuming than others. Come talk to me if you have no clue on how to proceed
		with this.
		\\ \\
		If you finished your tracker, and thought about how to evaluate it, then it is
		tiime to test your tracker on videos from different domains. For this, you can
		use the Internet to download videos or capture your own. For the conversion of
		your videos into frames, I suggest you use the linux-program mplayer. Matlab can
		also handle video-files, but you will get memory problems if you use videos that
		are longer than a couple of seconds.
		\\ \\
		Note that the most important part is to implement the tracker in a correct way.
		After you made sure your implementation is correct, try the tracker on a video
		in a different domain (search on the internet to find a good video or create your
		own video). Keep in mind that when looking for a suitable video, the goal is not
		to show how good the tracker works, but try to put your finger on the strong and
		weak parts of the algorithm. After you analyzed these strong and weak points of
		the algorithm, try to come up with suggestions to improve your design. If there
		is still enough time, you can implement one or more of your suggestions and analyze
		the results (for instance, how does your improvement compare to the original algorithm
		with respect to the weak as well as the strong parts of the original algorithm). If
		you run out of time, you can describe how you would change the design and explain your
		expectations of this change. 

	\section*{Conclusion}

	\begin{thebibliography}{2}
		\bibitem{KBOT}
			D. Comaniciu, V. Ramesh, P. Meer: \textit{Kernel-Based Object Tracking}
		\bibitem{COLOR}
			T. Gevers: \textit{Color in Image Search Engines}
	\end{thebibliography}

\end{document}
