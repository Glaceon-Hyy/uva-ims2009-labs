\documentclass[a4paper,11pt]{article}

\title{Intelligent Multimedia Systems \\ Example Questions}

\usepackage{amsmath}
\usepackage{graphicx}
\usepackage{float}
\newcommand{\ds}{\displaystyle}
\newcommand{\tbf}{\textbf}
\newcommand{\ol}{\overline}

\begin{document}
	\maketitle
	
	\section*{I. Exercise 1}
	
	\subsection*{a.} Draw light source S $(x=\frac{100}{300}, y=\frac{100}{300})$ and the color green with a line through the light source and the color green intersecting at 520nm in border.  Then draw a line from the green point towards the blue wavelength and in opposite direction to derive the color of the light source. This shows that a yellowish light source will give the green object a blue appearance.
	
	\subsection*{b.} Dependent on how the colors are plotted in the diagram. Saturation is defined by the distance from the center of the chromaticity diagram towards the edge (with edge the highest saturation). Artificial color = B > object color = K > sunlight = S.

	\subsection*{c.} 
		Plot the triangle.
	
	\subsection*{d.} 
		Not possible
	
	\section*{I. Exercise 2}
	
	\subsection*{a}
		Given is $A=500nm$, $X=0.0049$, $Y=0.323$, $Z=0.272$ \\
		Calculate the chromaticity coordinates $\ol{x}=\frac{X}{X+Y+Z}=\frac{0.0049}{0.0049+0.323+0.272}=0.008$, $\ol{y}=0.538$
	\subsection*{b}
		Plot point A at $520nm$ in diagram at $x=0.008$, $y=0.538$.
	\subsection*{c}
		Given is $A=580nm$, $X=0.0049$, $Y=0.323$, $Z=0.272$ \\
		Calculate the chromaticity coordinates $\ol{x}=\frac{X}{X+Y+Z}=\frac{0.9163}{0.9163+0.8700+0.0017}=0.5124$, $\ol{y}=0.4866$
	\subsection*{d}	
		Plot point B at $580nm$ in diagram at $x=0.5124$, $y=0.4866$.
	\subsection*{e}	
		$A=500nm$, $X=0.0049$, $Y=0.323$, $Z=0.272$\\
		$B=580nm$, $X=0.9163$, $Y=0.8700$, $Z=0.0017$\\ 
		color C, $X=0.9212$, $Y=1.1930$, $Z=0.2737$\\
		$\ol{x}=\frac{X_A+X_B}{X_A+X_B+Y_A+Y_B+Z_A+Z_B}=0.386$, $\ol{y}=0.5$, $\ol{z}=0.115$
	\subsection*{f}
		plot C at $\ol{x}=0.386$, $\ol{y}=0.5$
	\subsection*{g}
		If Intensity changes, $XYZ$ change proportionally (linearly), and the $xyz$ stay the same (nominator and denominator are changed proportionally, therefore the division is unchanged).
	\subsection*{h}	
		$\ol{x}=\frac{98.04}{98.04+100.00+118.12}=0.31$, $\ol{y}=0.316$, $\ol{z}=0.374$
	\subsection*{i}	
		Draw lines through the points A, B and C from point L. (A and B are already on the border, so whats the point here...)
	\subsection*{j}
		Line from L through C ends on $\lambda \approx 540$ based on the small plot (not very accurate).
	\subsection*{k}
		Colors A and B are on the border so have a maximum saturation, C is a mixture of both and therefore has a lower saturation.
	\subsection*{l}
		Draw lines from each point A B and C through L to the border. B and C are pure wavelenghts, A is not (it ends up on the border between 400nm and 700nm).
	\subsection*{m}
		Draw a nice triangle between the complementary points on the borders. 
	\subsection*{n}
		Estimate Hue from X-axis in the figure, and the Intensity from the Y-axis (not readable). Saturation = Hue - Intensity.	
	\subsection*{o}
		Read hue at the top of dominant wavelength. Read intensity from the amount of white light needed to create the color (white light is uniformly distributed in the spectral power distribution diagram). Saturation = Hue - Intensity.	
		
	
	\section*{I. Exercise 3}	
	\subsection*{a}
		$\theta=0$ between $\overrightarrow{n}$ and $\overrightarrow{l}$ is zero, therefore $cos(0)=1$. For any other $\theta$ the cosine will be smaller, the reflection of the incoming ray will get smaller.
	\subsection*{b}
		The R, G and B values will decrease linearly with respect to the decreasing intensity. Draw a nice cube with a line from (0,0,0) to (100,100,10)
		
	\subsection*{c}
		Question is ambigu.
		
	\subsection*{d}
	\[\frac{R}{G} = \frac{Ik_R \cos{\theta}}{Ik_G \cos{\theta}}=\frac{k_R}{k_G} \]		
	
	
	\section*{II. Exercise 1}
	\subsection*{a}
	\subsection*{b}
	\subsection*{c}
	\subsection*{d}
	\subsection*{e}
	In RGB, edge transitions and shadows will be classified by the edge detection. In rgb since it is invariant to illumination intensity, edge detection over the rgb image will return edges without shadows. Edges left out in rgb are shadows in RGB.
	\subsection*{f}
	Using rgb to detect edges and highlights, then classify the highlights using hue since hue is invariant to highlights.
	\subsection*{g}
	Take descriptors from the edges: edges form interest points in image from where to compute local features.
	\subsection*{h}
	local: robust against clutter and occlusion. \\
	global:	biased towards context, but more simple.
	
	\section*{II. Exercise 2}
	\subsection*{a} 
	if there are no white patches in the image, the resulting image will be incorrect.
	\subsection*{b} 
	white patch: the image will be rescaled towards pure white \\ 
	grey world assumption: the average image
	\subsection*{c}
	\subsection*{d}
	Average color difference is grey.
	\subsection*{e}
	A few colored regions.
	\subsection*{f}
	Weibull, based on texture and contrast. Look in the slides. (see lect9, slide 37)
	
	
	\section*{II. Exercise 6}
	\subsection*{a}
		Take the mean over Q1, R1 and R2 element-wise: $(0.4333, 0.5, 0.8, 0.2667)$
	\subsection*{b}
		Take the median over Q1, R1 and R2: $(0.4, 0.6, 0.8, 0.2)$
	\subsection*{c}
		If the user can be trusted, its safe to take the average. In the case of outliers, take the median.
		
	\section*{II. Exercise 7}	
	
	
\end{document}


